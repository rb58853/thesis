\begin{opinion}
El estudiante Raúl Beltrán Gómez desarrolló satisfactoriamente el trabajo de diploma titulado “Recuperación de imágenes usando incrustaciones multimodales”. En este trabajo el estudiante propuso el diseño e implementación de un prototipo de un sistema automatizado de extracción de características de imágenes para procesarlas, y un sistema de recuperación preciso que utiliza dichas caracteríticas como información para crear el proceso de recuperación, empleando modelos de aprendizaje automático.

El trabajo propone un acercamiento al problema de la recuperación precisa de imágenes. Para
ello se utiliza un nuevo enfoque: la aplicación de los modelos de \textit{Segmentación Segment Anything} (SAM) y el modelo \textit{Constractive Language-Image Pretraining} (CLIP) para la generación de \textit{embeddings} multimodales. Asimismo, se pone especial énfasis en la recuperación de imágenes mediante consultas precisas, para ello se tiene en cuenta la posición de segmentos de imágenes que componenen la misma, procesando tanto textos como imágenes. Esta investigación abre una nueva línea de investigación dentro del Grupo de Investigación en Inteligencia Artificial,

Para poder afrontar el trabajo, el estudiante tuvo que revisar literatura científica relacionada con la temática así como soluciones existentes y bibliotecas de \textit{software} que pueden ser apropiadas para su utilización. Todo ello con sentido crítico, determinando las mejores aproximaciones y también las dificultades que presentan.

Todo el trabajo fue realizado por la estudiante con una elevada constancia, capacidad de trabajo y habilidades, tanto de gestión, como de desarrollo y de investigación. 

Por estas razones pedimos que le sea otorgada al estudiante Raúl Beltrán Gómez la máxima calificación y, de esta manera, pueda obtener el título de Licenciado en Ciencia de la Computación.
\end{opinion}