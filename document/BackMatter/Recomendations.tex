\begin{recomendations}
    Para futuros trabajos, se recomienda implementar una serie de funciones para mejorar el comportamiento actual. El procesamiento del lenguaje en el trabajo se centra en los patrones de descripción de posiciones. Como mejora a esto, una propuesta sería preprocesar el texto para organizarlo de manera más adecuada para el cumplimiento de la tarea deseada. En esta misma fase de procesamiento de texto, otra posible implementación es el uso de modelos LLM (Large Language Models) para procesar la consulta y analizarla con un modelo tan potente como los que estarán disponibles en el futuro.
    
    En el campo del procesamiento de la imagen, se puede implementar otros reconocimientos de conceptos en la imagen, no solo reconocimiento posicional, sino que podría implementarse reconocimiento de colores, por ejemplo. Otro punto a destacar sería establecer parámetros y objetivos para ampliar este sistema, de esta forma, que no se centre solo en recuperación de texto a imagen, sino de imagen a imagen.
        
    Respecto a la experimentación, la concepción de una metodología experimental formal y exhaustiva representa una alternativa v\'alida. La elaboración de dicha metodología posibilitaría una comprensión más profunda y amplia sobre la eficacia de la propuesta en cuestión. Con el objetivo de establecer una metodología que permita la comparación automática y rigurosa de las distintas propuestas, resulta imperativo generar un conjunto de bases de datos diseñadas para evaluar variadas configuraciones de parámetros. La creación de múltiples bases de datos es imprescindible, dado que se cuenta con parámetros\supindex{}{Por ejemplo, los par\'ametros del entorno de SAM se usan en tiempo de indexación.} y propuestas que son variables el tiempo de indexación.

    Una vez almacenado este conjunto en el tiempo de indexación, se debe implementar un sistema de experimentación dinámico que permita la modificación de valores para cada selección de tipo de propuesta y parámetros en tiempo real\supindex{}{Por ejemplo, los par\'ametros del sistema de c\'alculo de similitud se usan en tiempo real.}. Además, se podría crear manualmente un conjunto de datos con sus valores esperados como salida para cada entradas.

\end{recomendations}
    