\begin{conclusions}
    En este estudio, se presentó una metodología para recuperar imágenes a partir de consultas en forma de lenguaje natural. Esta metodología se estructuró en tres etapas clave: procesamiento de imágenes, procesamiento de consultas y recuperación de información. Se puso especial énfasis en el uso de embeddings y posiciones como características más relevantes.
    
    Los resultados de esta metodología fueron satisfactorios, logrando una notable mejora en la relevancia al utilizar consultas más precisas que indican relaciones posicionales de la imagen. Se llegó a la conclusión de que el uso de características como los embeddings generados por un lenguaje multimodal permite una adecuada recuperación de las imágenes en términos de comprensión de conceptos. La combinación de estas características con la detección de objetos tanto en la imagen como en el lenguaje natural logran un buen resultado para el objetivo de recuperar imágenes a partir de consultas precisas.
    \end{conclusions}
    
    